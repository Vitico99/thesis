\begin{conclusions}
    % El objetivo fundamental de este trabajo fue realizar una primera
    % aproximaci\'on a la recomendaci\'on de configuraciones
    % gr\'aficas mediante aprendizaje de m\'aquinas.



   
    % A partir de la profundizaci\'on en el estado del arte de la representaci\'on
    % computacional de visualizaciones y la recomendaci\'on de configuraciones gr\'aficas, se desarroll\'o un marco
    % de trabajo abstracto para modelar la recomendaci\'on de configuraciones gr\'aficas,
    % mediante enfoques de aprendizaje de m\'aquinas sobre grafos. Se concibi\'o
    % una propuesta de sistema basado en el marco de trabajo propuesto, sobre cuya base se implement\'o un prototipo funcional.

    % La adaptaci\'on de m\'etodos transductivos de \textit{knowledge graph embedding} 
    % al contexto de la recomendaci\'on de visualizaciones, a
    % trav\'es de la metaheur\'istica de universos paralelos, constituye el 
    % aporte de este trabajo al campo de la visualizaci\'on inteligente autom\'atica de datos.
    


    En este trabajo se desarroll\'o un sistema de recomendaci\'on de configuraciones gr\'aficas
    basado en aprendizaje de m\'aquinas sobre grafos.

    Para lograr este objetivo se realiz\'o un estudio sobre el estado del arte de la representaci\'on
    computacional de visualizaciones y de los sistemas de recomendaci\'on de configuraciones gr\'aficas. En este se analizaron
    las caracter\'isticas principales de los lenguajes de visualizaci\'on de datos y los enfoques de los sistemas
    desarrollados por previas investigaciones.
    
    A partir de la profundizaci\'on en el estado del arte se plante\'o un marco de trabajo abstracto
    para modelar la recomendaci\'on de configuraciones gr\'aficas. En dicho marco se
    representaron los conjuntos de datos mediante grafos de conocimientos
    y se definieron dos tareas de aprendizaje de m\'aquinas sobre grafos, cuyos resultados permiten comparar las
    posibles visualizaciones de un conjunto de datos.

    Se concibi\'o una propuesta de sistema fundamentada en las ideas y definiciones presentadas en el marco
    de trabajo presentado. Para solucionar las tareas de aprendizaje de m\'aquinas planteadas se propusieron
    los modelos PuLinkPrediction y PuNodeClassification. Estos modelos adaptan
    los algoritmos transductivos de \textit{Knowledge Graph Embedding} (KGE)
    al contexto de la recomendaci\'on de configuraciones gr\'aficas mediante la metaheur\'istica de universos paralelos.


    Sobre la base del sistema dise\~nado se implement\'o un prototipo funcional que fue evaluado mediante la realizaci\'on
    de una serie de experimentos. Los resultados obtenidos evidenciaron que el prototipo desarrollado recomend\'o
    la visualizaci\'on esperada en su top 5 un 66\% de las veces, lo que permiti\'o validar la propuesta realizada. 
    Adem\'as, se expusieron las limitaciones de rendimiento del sistema y se plantearon un conjunto de factores que pudieran influir en los resultados
    del mismo.



\end{conclusions}
