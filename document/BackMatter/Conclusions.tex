\begin{conclusions}
    El objetivo fundamental de este trabajo fue realizar una primera
    aproximaci\'on a la recomendaci\'on de configuraciones
    gr\'aficas mediante aprendizaje de m\'aquinas.
   
    A partir de la profundizaci\'on en el estado del arte de la representaci\'on
    computacional de visualizaciones y la recomendaci\'on de configuraciones gr\'aficas, se desarroll\'o un marco
    de trabajo abstracto para modelar la recomendaci\'on de configuraciones gr\'aficas,
    mediante enfoques de aprendizaje de m\'aquinas sobre grafos. Se concibi\'o
    una propuesta de sistema basado en el marco de trabajo propuesto, sobre cuya base se implement\'o un prototipo funcional.

    La adaptaci\'on de m\'etodos transductivos de \textit{knowledge graph embedding} 
    al contexto de la recomendaci\'on de visualizaciones, a
    trav\'es de la metaheur\'istica de universos paralelos, constituye el 
    aporte de este trabajo al campo de la visualizaci\'on inteligente autom\'atica de datos.
    
    Se ejecutaron experimentos que permitieron establecer
    la validez de la propuesta realizada mediante la evaluaci\'on del prototipo y
    se plante\'o un conjunto de factores que pudieran influir en los resultados
    del sistema. 
\end{conclusions}
