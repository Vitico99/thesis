\begin{recomendations}
    A partir de los desaf\'ios computacionales encontrados, as\'i como de los
    resultados obtenidos por el sistema desarrollado, se identifican nuevas l\'ineas
    de investigaci\'on que permitan mejorar la efectividad de los sistemas de
    recomendaci\'on de configuraciones gr\'aficas.

    \begin{itemize}
        \item La creaci\'on de corpus de dominios espec\'ificos que permitan
        evaluar el efecto de la heterogeneidad de los dominios sobre los sistemas
        de recomendaci\'on de visualizaciones.
        \item Los modelos implementados utilizan peque\~nos subgrafos del grafo de
        entrenamiento para realizar las predicciones, por lo que ser\'ia posible
        almacenar el grafo de entrenamiento en almacenamiento externo. Para esto se propone la
        implementaci\'on del sistema utilizando bases de datos orientadas a grafos.
        \item Incluir relaciones entre atributos mediante funciones multivariable utilizadas por sistemas 
        basados en reglas.
        \item Realizar estudios sobre las capacidades del sistema para incorporar preferencias de usuario
        mediante la adici\'on din\'amica de entidades y relaciones al grafo de entrenamiento.
    \end{itemize}


\end{recomendations}
