\chapter*{Introducción}\label{chapter:introduction}

El ascenso de la era de la informaci\'on 
caracterizado por la r\'apida expansi\'on
de Internet y la adopci\'on generalizada de las computadoras
personales ha provocado que una gran variedad de datos sean producidos
en un volumen y velocidad cada vez mayores. Este fen\'omeno denominado
Big Data ha tenido un gran impacto en el desarrollo de distintas \'areas de la actividad humana como la ciencia y la industria, ya que mediante el procesamiento de estas enormes colecciones de datos se obtiene informaci\'on actualizada y relevante que permite tomar decisiones de forma r\'apida y segura. Debido a los beneficios potenciales que presenta el Big Data ha sido necesario el surgimiento y evoluci\'on de tecnolog\'ias y procesos que permitan su utilizaci\'on, siendo el an\'alisis de datos de los procesos m\'as relevantes dentro de este ecosistema dado que durante el mismo se realiza el descubrimiento de informaci\'on.

Dentro del an\'alisis de datos existen m\'ultiples enfoques y facetas, resultando de particular inter\'es el an\'alisis exploratorio el cual utiliza estad\'istica descriptiva y visualizaciones para descubrir las caracter\'isticas principales de un conjunto de datos permitiendo generar hip\'otesis. En la actualidad este enfoque requiere de especialistas del dominio con profundos conocimientos t\'ecnicos y supone un gasto considerable de tiempo y esfuerzo debido al tama\~no y dimensi\'on de los conjuntos de datos y la ausencia de herramientas que faciliten un an\'alisis visual m\'as r\'apido.
Las herramientas actuales de visualizaci\'on de datos como Excel, Tableau y Spotfire ofrecen facilidades para la especificaci\'on manual de visualizaciones, sin embargo, carecen de la capacidad de guiar al usuario a trav\'es del espacio de posibles visualizaciones que se pueden especificar por lo cual el usuario debe de buscar a ciegas en este espacio hasta encontrar alguna que sea relevante, siendo esto un procedimiento tedioso con numerosos intentos y errores. 

Estas limitaciones fueron recogidas y analizadas por Vartak et al. % Add the biblio reference here
(2016) definiendo y proponiendo como soluci\'on los sistemas de recomendaci\'on de visualizaciones (VizRec) % Change the format of VizRec to match the paper
los cuales tienen como objetivo construir y recomendar de forma autom\'atica visualizaciones que resalten patrones o tendencias de inter\'es, permitiendo un an\'alisis visual m\'as r\'apido. % quotes here?
Esta definici\'on depende del concepto humano de \textit{inter\'es} el cual es subjetivo por lo que dicho art\'iculo plantea cinco ejes (caracter\'isticas) para definir \textit{inter\'es} los cuales se denominaron \textit{ejes de recomendaci\'on}. A partir de la publicaci\'on de este art\'iculo se han realizado varias implementaciones de este tipo de sistemas resolviendo distintas combinaciones de los ejes mediante diferentes enfoques computacionales.

[Definir los ejes que se van a resolver en este trabajo y hablar de las limitaciones de las implementaciones que han intentado resolver esos ejes?]

[Explicar como mi propuesta solucionar\'ia estas limitaciones o mejorar\'ia algo de lo que se ha hecho]

[Hablar de LETO]

\addcontentsline{toc}{chapter}{Introducción}
