\chapter*{Introducción}\label{chapter:introduction}

Desde los a\~nos finales del siglo XX, la
r\'apida expansi\'on de Internet y la adopci\'on generalizada de las
computadoras personales ha provocado que una gran variedad de datos sea
producida en un volumen y velocidad cada vez mayores. Este fen\'omeno
denominado ``Big Data'' \cite{beyer2012importance} ha tenido un gran impacto en distintas \'areas de
la actividad humana, permitiendo el desarrollo de soluciones adaptables a los
requerimientos de diversos dominios de la ciencia y organizaciones industriales.

Mediante el procesamiento de estas enormes colecciones de datos se
obtiene informaci\'on, actualizada y relevante, que puede ser utilizada para
formular nuevas hip\'otesis o transformarse en una ventaja competitiva, permitiendo
una toma de decisiones m\'as r\'apida y segura \cite{de2016formal}. Los beneficios
potenciales del Big Data han propiciado el surgimiento y desarrollo de tecnolog\'ias
y procesos que permitan su utilizaci\'on, siendo el an\'alisis de datos de los procesos
m\'as relevantes dentro de este ecosistema.

Dentro del an\'alisis de datos existen m\'ultiples enfoques y m\'etodos, resultando
de particular inter\'es para este trabajo la visualizaci\'on de datos. La capacidad
de las representaciones visuales de presentar informaci\'on, de forma intuitiva para su
f\'acil comprensi\'on y an\'alisis, ha llevado al auge de esta rama en los campos
de la Estad\'istica y la Comunicaci\'on.

El t\'ermino visualizaci\'on de datos tiene una larga historia que se remonta al
siglo II AC con la aparici\'on de los primeros mapas y cartas de navegaci\'on. La invenci\'on de las computadoras provoc\'o un cambio enorme
en las formas visuales de representar datos. En particular, el desarrollo de \textit{software} ha provocado
un gran avance en la aplicaci\'on de la visualizaci\'on de datos, permitiendo a los usuarios
manipular cantidades sustanciales de estos de forma intuitiva y construir
visualizaciones de una forma m\'as r\'apida y efectiva \cite{li2020overview}. 

En la actualidad, la visualizaci\'on de datos
es usualmente asociada con la Ciencia de la Computaci\'on, por lo que la mayor\'ia
de sus definiciones se enfocan en la conexi\'on entre los datos y la tecnolog\'ia
computacional que los representa visualmente 
[\cite*{card1999readings}, \cite*{friendly2001milestones}, \cite*{manovich2010visualization}, \cite*{kirk2012data}].
En la presente investigaci\'on se utilizar\'a la definici\'on dada por
Bikakis \cite{bikakis2018big} donde se plantea lo siguiente:\\
\begin{quotation}
    \textit{``La visualizaci\'on de datos es la representación de datos en un
    formato pictórico o gráfico, y una herramienta de visualización de
    datos es el software que genera esta representación. La visualización
    de datos proporciona a los usuarios un medio intuitivo para identificar patrones
    interesantes, inferir correlaciones y causalidades de manera efectiva,
    respaldando el proceso de toma de decisiones.''}
\end{quotation}


\section*{Motivaci\'on}


La visualizaci\'on de datos es uno de los m\'etodos
anal\'iticos m\'as utilizados hoy en d\'ia. Este requiere de especialistas
del dominio con profundos conocimientos t\'ecnicos, suponiendo un gasto
considerable de tiempo y esfuerzo humano debido al tama\~no y dimensi\'on
de los conjuntos de datos y a la ausencia de herramientas que faciliten un
an\'alisis visual m\'as r\'apido [\cite*{chen2012business}, \cite*{vartak2017towards}].

Dichas limitaciones han sido recogidas y analizadas en varios trabajos dentro de la
literatura [\cite*{vartak2017towards}, \cite*{zeng2021we}, \cite*{godfrey2016interactive}], 
dando lugar a la definici\'on de caracter\'isticas, dificultades y 
directrices del problema de visualizaci\'on autom\'atica de datos.
Vartak y col. \cite{vartak2017towards} definieron los sistemas de recomendaci\'on
de visualizaciones (VizRec) como respuesta a este problema. Estos sistemas tienen como
objetivo construir y sugerir de forma autom\'atica visualizaciones que resalten
patrones o tendencias de inter\'es, permitiendo un an\'alisis visual m\'as r\'apido
y efectivo.

La publicaci\'on de este art\'iculo ha despertado un renovado inter\'es en esta \'area de
investigaci\'on, surgiendo varios prototipos con nuevos enfoques que intentan solucionar
el problema de forma total o parcial 
[\cite*{luo2018deepeye}, \cite*{moritz2018draco}, \cite*{dibia2019data2vis}, \cite*{hu2019vizml}, \cite*{li2021kg4vis}, \cite*{harris2021insight}].
Uno de los subproblemas que han intentado resolver varios sistemas es la recomendaci\'on de configuraciones gr\'aficas, utilizando enfoques
basados en reglas y enfoques basados en aprendizaje de m\'aquinas (ML por sus siglas en ingl\'es) \cite{zeng2021we}. 
Este \'ultimo enfoque ha surgido recientemente, por lo que existen pocos trabajos al
respecto, y no se han explorado todas las diversas t\'ecnicas de modelaci\'on de problemas
de aprendizaje de m\'aquinas aplicadas al contexto de la recomendaci\'on de configuraciones gr\'aficas.


\section*{Antecedentes}
En la facultad de Matem\'aticas y Computaci\'on de la Universidad de la Habana,
investigadores del grupo de Inteligencia Artificial han desarrollado una l\'inea
de investigaci\'on centrada en la visualizaci\'on autom\'atica de datos.

Dentro de esta l\'inea de investigaci\'on surge el sistema \textbf{Learning Engine
Through Ontologies} (LETO) \cite{estevez2019demo}, un marco de trabajo para el descubrimiento
de informaci\'on relevante a partir de datos estructurados y no estructurados. Este
sistema permite generar conocimiento mediante la creaci\'on y composici\'on de ontolog\'ias 
representadas utilizando grafos de conocimientos. Una de las l\'ineas
de trabajo futuro sugerida por los autores del sistema fue la
investigaci\'on sobre la utilizaci\'on de t\'ecnicas
de aprendizaje de m\'aquinas para la generaci\'on de configuraciones visuales, ya que
dicho sistema utilizaba reglas para la soluci\'on de este problema.

Esta investigaci\'on aborda la l\'inea propuesta, inspir\'andose en el trabajo realizado
por el sistema desarrollado en la entidad, presentando una primera aproximaci\'on a la generaci\'on 
de configuraciones gr\'aficas con la utilizaci\'on de t\'ecnicas de aprendizaje de m\'aquinas sobre grafos
de conocimientos.


\section*{Problem\'atica}

El dise\~no de visualizaciones es un proceso complejo 
donde se deben de tener en consideraci\'on m\'ultiples factores y no existe un
consenso, incluso, entre los expertos del dominio.

En la literatura se encuentran varios trabajos donde se se\~nala la importancia
de los elementos del dise\~no gr\'afico,
de los objetivos del an\'alisis, y de las
preferencias personales de los usuarios en la construcci\'on de visualizaciones \cite{zeng2021we}. Adem\'as,
se espera que los sistemas de recomendaci\'on de configuraciones
gr\'aficas sugieran m\'ultiples visualizaciones, por lo que
la calidad del conjunto de visualizaciones recomendado es tan
importante como la calidad individual de sus elementos. Criterios
como la diversidad de las recomendaciones y el cubrimiento
del espacio de posibles visualizaciones han sido aspectos
espec\'ificos de este problema destacados en la literatura \cite{hu2019vizml}.


Debido a los motivos anteriores, la automatizaci\'on del proceso de recomendaci\'on de un gr\'afico, mediante
la elecci\'on de configuraciones gr\'aficas determinadas, es una tarea compleja para
la que a\'un no existe una soluci\'on universalmente aceptada.

El aprendizaje de m\'aquinas es una disciplina que ha destacado en la
resoluci\'on de tareas complejas, por lo que resulta de inter\'es
el estudio y desarrollo de sistemas de recomendaci\'on de configuraciones
gr\'aficas basados en t\'ecnicas de aprendizaje de m\'aquinas.


\section*{Objetivos}

\subsubsection{Objtivo general}
Proponer y desarrollar un sistema de recomendaci\'on de configuraciones
visuales basado en aprendizaje de m\'aquinas.

\subsubsection{Objetivos espec\'ificos}
\begin{enumerate}
   \item Estudiar diferentes sistemas de recomendaci\'on de configuraciones
   gr\'aficas as\'i como sus enfoques para resolver la problem\'atica presentada.
    
    \item Modelar formalmente el problema de recomendaci\'on de configuraciones
    gr\'aficas.

    \item Dise\~nar y desarrollar un prototipo de sistema que sugiera configuraciones
    gr\'aficas a partir de un conjunto de datos preseleccionado.

    \item Realizar experimentos que permitan evaluar la validez de la propuesta
    concebida y establecer pautas para el desarrollo a seguir.

\end{enumerate}

\section*{Propuesta de soluci\'on}

Como m\'etodo de soluci\'on se propone un
sistema que modela el proceso de construcci\'on
de una visualizaci\'on a trav\'es de la codificaci\'on
individual de cada uno de los atributos del conjunto de datos. Los conjuntos
de datos son representados mediante grafos que modelan
las relaciones establecidas entre los atributos de los
conjuntos y sus caracter\'isticas. Posteriormente,
este grafo es utilizado como entrada de modelos
de \textit{Graph Representation Learning} que calculan las probabilides
de utilizar determinadas opciones gr\'aficas. Estas probabilidades
son combinadas para calcular qu\'e tan probable es que un conjunto de
datos sea representado por un gr\'afico determinado. Dicho valor
es utilizado para comparar los elementos del espacio de posibles
visualizaciones y realizar recomendaciones.


\section*{Estructura del documento}

El resto del documento se ha estructurado
en cuatro cap\'itulos que abordan las distintas fases
por las que transit\'o la presente investigaci\'on. En el cap\'itulo \ref{chapter:state-of-the-art} 
se realiza un estudio sobre el estado del arte de la recomendaci\'on
de configuraciones visuales, presentando varios conceptos y enfoques
aplicados en cap\'itulos posteriores. El cap\'itulo \ref{chapter:ml-on-graphs}
introduce un marco de trabajo para la modelaci\'on del problema de recomendaci\'on
de configuraciones gr\'aficas como un problema de aprendizaje de m\'aquinas sobre grafos.
El cap\'itulo \ref{chapter:proposal} presenta una propuesta de sistema de recomendaci\'on
de configuraciones gr\'aficas utilizando los conceptos y problemas definidos en el cap\'itulo anterior.
En el cap\'itulo \ref{chapter:implementation} se detallan los
aspectos t\'ecnicos de la implementaci\'on de un prototipo del sistema y
se realiza un an\'alisis de la validez de la soluci\'on implementada
mediante el desarrollo de experimentos. Como parte del desenlace
se presentan las conclusiones, que recogen los resultados
obtenidos de acuerdo al cumplimiento de los objetivos propuestos, as\'i como 
las recomendaciones, donde se proponen un conjunto de l\'ineas
de investigaci\'on como parte de la continuaci\'on del presente trabajo.
Finalmente, se presentan las referencias bibliogr\'aficas que sustentan
la base cient\'ifica de la soluci\'on propuesta.





\addcontentsline{toc}{chapter}{Introducción}
