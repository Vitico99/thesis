\chapter*{Introducción}\label{chapter:introduction}

Durante los a\~nos finales del siglo XX y principios del siglo XXI la
r\'apida expansi\'on de Internet y la adopci\'on generalizada de las
computadoras personales ha provocado que una gran variedad de datos sea
producida en un volumen y velocidad cada vez mayores. Este fen\'omeno
denominado ``Big Data'' \cite{beyer2012importance} ha tenido un gran impacto en distintas \'areas de
la actividad humana permitiendo el desarrollo de soluciones adaptables a los
requerimientos de diversos dominios de la ciencia y organizaciones industriales.

Mediante el procesamiento de estas enormes colecciones de datos se
obtiene informaci\'on actualizada y relevante que puede ser utilizada para
formular nuevas hip\'otesis o transformarse en una ventaja competitiva permitiendo
una toma de decisiones m\'as r\'apida y segura \cite{de2016formal}. Debido a los beneficios potenciales
que presenta el Big Data ha sido necesario el surgimiento y evoluci\'on de
tecnolog\'ias y procesos que permitan su utilizaci\'on, siendo el an\'alisis
de datos de los procesos m\'as relevantes dentro de este ecosistema.

Dentro del an\'alisis de datos existen m\'ultiples enfoques y m\'etodos, resultando
de particular inter\'es para este trabajo la visualizaci\'on de datos. Esta rama se enmarca
dentro del campo de la Estad\'istica como parte del an\'alisis exploratorio de datos.
El t\'ermino visualizaci\'on de datos tiene una larga historia que se remonta al
siglo II AC con la aparici\'on de los primeros mapas y cartas de navegaci\'on utilizadas
para explorar el mundo. La invenci\'on de las computadoras provoc\'o un cambio enorme
en las formas visuales de representar datos. En particular, el desarrollo de \textit{software} ha provocado
un gran avance en la aplicaci\'on de la visualizaci\'on de datos, permitiendo a los usuarios
manipular cantidades sustanciales de datos de forma intuitiva y construir
visualizaciones de una forma m\'as r\'apida y efectiva \cite{li2020overview}. 

En la actualidad la visualizaci\'on de datos
es usualmente asociada con la Ciencia de la Computaci\'on, por lo que la mayor\'ia
de sus definiciones se enfocan en la conexi\'on entre los datos y la t\'ecnolog\'ia
computacional que los representa visualmente 
[\cite*{card1999readings}, \cite*{friendly2001milestones}, \cite*{manovich2010visualization}, \cite*{kirk2012data}].
En la presente investigaci\'on se utilizar\'a la definici\'on provista por
Bikakis \cite{bikakis2018big} donde se plantea lo siguiente:\\
\begin{quotation}
    \textit{``La visualizaci\'on de datos es la representación de datos en un
    formato pictórico o gráfico, y una herramienta de visualización de
    datos es el software que genera esta representación. La visualización
    de datos proporciona a los usuarios un medio intuitivo para identificar patrones
    interesantes, inferir correlaciones y causalidades de manera efectiva,
    respaldando el proceso de toma de decisiones.''}
\end{quotation}


\section*{Motivaci\'on}

El dise\~no de visualizaciones puede resultar ser un proceso complejo 
donde se deben de tener en consideraci\'on m\'ultiples factores. 
En la literatura se encuentran varios trabajos los cuales se\~nalan la importancia
de los elementos del dise\~no gr\'afico,
de los objetivos o la tarea a realizar dentro del an\'alisis, e incluso, de las
preferencias personales de los usuarios \cite{zeng2021we}.

En la actualidad la visualizaci\'on de datos es uno de los m\'etodos
anal\'iticos m\'as utilizados. Este requiere de especialistas
del dominio con profundos conocimientos t\'ecnicos, suponiendo un gasto
considerable de tiempo y esfuerzo humano debido al tama\~no y dimensi\'on
de los conjuntos de datos y la ausencia de herramientas que faciliten un
an\'alisis visual m\'as r\'apido [\cite*{chen2012business}, \cite*{vartak2017towards}].

Dichas limitaciones han sido recogidas y analizadas por varios trabajos dentro de la
literatura [\cite*{zeng2021we}, \cite*{vartak2017towards}, \cite*{godfrey2016interactive}], 
dando lugar a la definici\'on de caracter\'isticas, dificultades y 
directrices del problema de visualizaci\'on autom\'atica de datos.
Vartak y col. \cite{vartak2017towards} definieron los sistemas de recomendaci\'on
de visualizaciones (VizRec) como respuesta a este problema. Estos sistemas tienen como
objetivo construir y sugerir de forma autom\'atica visualizaciones que resalten
patrones o tendencias de inter'es, permitiendo un an\'alisis visual m\'as r\'apido
y efectivo.

La publicaci\'on de este art\'iculo ha despertado un renovado inter\'es en esta \'area de
investigaci\'on, surgiendo varios prototipos con nuevos enfoques que intentaban solucionar
el problema de forma total o parcial 
[\cite*{luo2018deepeye}, \cite*{moritz2018draco}, \cite*{dibia2019data2vis}, \cite*{hu2019vizml}, \cite*{li2021kg4vis}, \cite*{harris2021insight}].
Uno de los subproblemas que han intentando resolver varios sistemas es el problema
de recomendaci\'on de configuraciones gr\'aficas, utiliz\'andose enfoques
basados en reglas y enfoques basados en aprendizaje de m\'aquinas (ML por sus siglas en ingl\'es) \cite{zeng2021we}. 
Este u\'ltimo enfoque ha surgido recientemente por lo que existen pocos trabajos al
respecto y no se han explorado todas las diversas t\'ecnicas de modelaci\'on de problemas
de aprendizaje de m\'aquinas aplicadas al contexto de la recomendaci\'on de configuraciones gr\'aficas.


\section*{Antecedentes}
En la facultad de Matem\'aticas y Computaci\'on de la Universidad de la Habana,
investigadores del grupo de Inteligencia Artificial han desarrollado una l\'inea
de investigaci\'on centrada en la visualizaci\'on autom\'atica de datos.

Dentro de esta l\'inea de investigaci\'on surge el sistema \textbf{Learning Engine
Through Ontologies} (LETO) \cite{estevez2019demo}, un marco de trabajo para el descubrimiento
de informaci\'on relevante a partir de datos estructurados y no estructurados. Este
sistema permite generar conocimiento mediante la creaci\'on y composici\'on de ontolog\'ias 
representadas utilizando grafos de conocimientos. Una de las l\'ineas
de trabajo futuro sugerida por los autores del sistema fue la
investigaci\'on sobre la utilizaci\'on de t\'ecnicas
de aprendizaje de m\'aquinas para la generaci\'on de configuraciones visuales, ya que
este sistema utilizaba reglas para la soluci\'on este problema.

Esta investigaci\'on aborda la l\'inea propuesta, inspir\'andose en el trabajo realizado
por el sistema desarrollado en la entidad, presentando una primera aproximaci\'on a la generaci\'on 
de configuraciones gr\'aficas con la utilizaci\'on de t\'ecnicas de aprendizaje de profundo sobre grafos.


\section*{Problem\'atica}

Gran parte de los sistemas que conforman el estado del arte utilizan
un enfoque basado en reglas para resolver el problema de recomendaci\'on
de configuraciones gr\'aficas (Cap\'itulo \ref{chapter:state-of-the-art}).
Este tipo de enfoques son dif\'iciles de implementar y extender debido
a que para definir reglas se necesita de expertos que las
definan manualmente, resultando en largos tiempos de desarrollo para sistemas complejos.

El aprendizaje de m\'aquinas es un nuevo enfoque que ha surgido
recientemente por lo que existen muy pocos trabajos al respecto. 
En particular,
el aprendizaje de m\'aquinas sobre grafos solo ha sido utilizado por un
sistema. 
Los grafos son estructuras que permiten la modelaci\'on
de problemas complejos y los
sistemas de recomendaci\'on basados en aprendizaje de m\'aquinas
sobre grafos han sido extensamente utilizados en varios tipos de problemas con buenos resultados \cite{guo2020survey}.
Debido a esto se considera necesario profundizar en la viabilidad de la utilizaci\'on
de este tipo de sistemas para la recomendaci\'on de configuraciones gr\'aficas.
\section*{Objetivos}

\subsubsection{Objtivo general}
Proponer y desarrollar un sistema de recomendaci\'on de configuraciones
visuales basado en la utilizaci\'on de t\'ecnicas de aprendizaje profundo
sobre grafos.

\subsubsection{Objetivos espec\'ificos}
\begin{enumerate}
   \item Estudiar diferentes sistemas de recomendaci\'on de configuraciones
   gr\'aficas as\'i como sus enfoques para resolver la problem\'atica presentada.
    
    \item Definir formalmente el problema de generaci\'on de configuraciones
    gr\'aficas como un problema de aprendizaje de m\'aquinas sobre grafos.

    \item Dise\~nar y desarrollar un prototipo de sistema basado en aprendizaje de m\'aquinas sobre grafos que permita generar configuraciones
    gr\'aficas a partir de un conjunto de datos preseleccionado.

    \item Evaluar la propuesta presentada y compararla con otros sistemas de
    recomendaci\'on de configuraciones gr\'aficas.

\end{enumerate}



\addcontentsline{toc}{chapter}{Introducción}
