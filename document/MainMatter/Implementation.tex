\chapter{Detalles de Implementación y Experimentos}\label{chapter:implementation}


\section{Herramientas y tecnolog\'ias utilizadas}

\subsection{Lenguaje de programaci\'on Python}

Python es un lenguaje de programaci\'on de alto nivel y de prop\'osito
general. Es interpretado, multi-paradigma, de tipado din\'amico y memoria
auto-manejada. Fue desarrollado por Guido Van Rossum en 1991 y la \'ultima
versi\'on al momento de realizarse este trabajo es la (incluirversion).
La mayor\'ia de las implementaciones incluyen un bucle \textit{Lectura-Evaluaci\'on-Impresi\'on}
(REPL por sus siglas en ingl\'es) funcionando como un int\'erprete de l\'ineas
de comandos, donde el usuario introduce las sentencias secuencialmente y recibe
inmediatamente los resultados.

En octubre de 2021 alcanz\'o el primer puesto en la lista de los lenguajes
de programaci\'on m\'as populares seg\'un el \'indice de la comunidad de
programaci\'on TIOBE 2021. Es utilizado por grandes organizaciones como
Google [PSF2021b], Yahoo[PSF2020], la NASA[PSF2021c], el CERN [CERN 2014],
Wikipedia, Amazon, Facebook, Instagram, Spotify, entre otros.

Es muy usado para el an\'alisis e ingenier\'ia de datos, aprendizaje
por computadoras e inteligencia artificial gracias a diversas bibliotecas
como NumPy, SciPy, Pandas, Matplotlib, Tensorflow, Keras, Pytorch, OpenCV y Networkx.
Para desarrollo web cuenta con marcos de trabajo como FastAPI, Flask y Django.
Adem\'as, se emplea en otros campos como la rob\'otica, el internet de las
cosas y la educaci\'on.
Para este trabajo se utiliz\'o la versi\'on 3.7 para lograr una retro-compatibilidad
alta.


\subsection{Pandas}
Pandas es una biblioteca de c\'odigo abierto del lenguaje Python orientada
a la manipulaci\'on y an\'alisis de datos. Su implmentaci\'on es
r\'apida, poderosa y f\'acil de usar lo que la ha convertido en
una herramienta muy popular entre los analistas de datos en dominios 
acad\'emicos y comerciales \cite{pandas2022}.
Esta biblioteca provee objetos para la manipulaci\'on r\'apida
y eficiente de conjuntos de datos, es compatible con m\'ultiples formatos
de almacenamiento de datos e implementa operaciones optimizadas para
modificar conjuntos de datos.

En este modelo Pandas es utilizada para realizar inferencia de tipos
sobre los atributos de los conjuntos de datos y fue fundamental para
la creaci\'on de los conjuntos de entrenamiento del modelo.

\subsection{Numpy}
Numpy es una biblioteca de c\'odigo abierto para el lenguaje de
programaci\'on Python que permite crear vectores y matrices de gran
tama\~no y de muchas dimensiones. Adem\'as, incluye
funciones para operar con sus matrices y vectores de una forma
c\'omoda, siendo comparable a Matlab en ese sentido. Internamente
utiliza el lenguaje C para la implementaci\'on de las funciones, por
lo que brinda un alto rendimiento \cite{harris2020array}.

Este trabajo utiliza las implementaciones eficiente de arreglos y m\'etodos estad\'isticos
de esta biblioteca dentro del proceso de extracci\'on de vectores de caracter\'isticas.

\subsection{SciPy}
SciPy es una biblioteca de c\'odigo abierto para el lenguaje de
programaci\'on Python la cual est\'a enfocada en proveer algoritmos
para problemas de optimizaci\'on, integraci\'on, interpolaci\'on,
c\'alculo de valores propios, ecuaciones algebraicas, ecuaciones diferenciales,
estad\'isticas entre otros. Gracias a la utilizaci\'on de estructuras
especializadas para el c\'omputo sobre arreglos y sus implementaciones
altamente eficientes escritas en lenguajes de bajo nivel como Fortran, C y C++
permite resolver problemas complejos en utilizando una sintaxis de alto nivel a la
vez que r\'apida y eficiente \cite{2020SciPy-NMeth}.

SciPy es utilizada dentro del modelo durante la extracci\'on de vectores
de caracter\'isticas para realizar el c\'omputo de m\'etricas estad\'isticas complejas.

\subsection{Scikit-Learn}
Scikit-Learn es una biblioteca de aprendizaje de m\'aquinas de c\'odigo abierto para el
lenguaje de programaci\'on Python. Presenta una gran colecci\'on de modelos
para la resoluci\'on de problemas enmarcados dentro del paradigma de aprendizaje de m\'aquinas.
En la actualidad incluye modelos para problemas de clasificaci\'on, regresi\'on, 
clusterizaci\'on, reducci\'on de dimensionalidad y optimizaci\'on de modelos \cite{scikit-learn}.

Varios modelos utilizados como punto de referencia para comparar los resultados
de los experimentos realizados han sido provistos por esta biblioteca, adem\'as se han
empleado sus m\'etodos de preprocesamiento para la preparaci\'on del corpus.

\subsection{Tensorflow y Keras}
Tensorflow es una plataforma de c\'odigo abierto desarrollada por Google para
aprendizaje por computadoras de extremo a extremo. Proporciona interfaces para
C++, Haskell, Java, Go, Rust y Python de forma oficial, adem\'as de interfaces de
terceros para C\#, Julia, R y Scala. Su implementaci\'on permite ejecutarse
eficientemente tanto sobre CPU como GPU y TPU \cite{tensorflow2015-whitepaper}.

Keras es una interfaz de alto nivel de Tensorflow que permite resolver problemas
de aprendizaje autom\'atico con un enfoque en el aprendizaje profundo de una forma
accesible y altamente productiva \cite{chollet2015keras}.

Tensoflow y Keras fueron utilizados para la implementaci\'on y entrenamiento 
del modelo de clasificaci\'on de nodos y en el modelo de clasificaci\'on utilizado
como punto de referencia.

\subsection{Google Colaboratory}
Es un servicio gratuito de Google que permite escribir y ejecutar
c\'odigo arbitrario de Python sobre recursos computacionales como GPU y TPU en
el navegador. Esta plataforma ofrece un entorno de trabajo de Python con buenos recursos computacionales,
bibliotecas y herramientas de aprendizaje autom\'atico pre-instaladas de forma
gratuita (tambi\'en se ofrece la posibilidad de aumentar los recursos computaciones del entorno mediante una suscripci\'on de pago).
Basada en Jupyter Notebook brinda en su plan gratuito 13 GB de RAM, 110GB de almacenamiento (extensible mediante la utilizaci\'on de Google Drive) y
la posibilidad de utilizar tanto CPU como GPU y TPU \cite{google2014colab}.

Este servicio fue utilizado para la creaci\'on del corpus, entrenamiento y evaluaci\'on
del modelo.







