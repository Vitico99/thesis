\begin{opinion}
    El estudiante Victor Manuel Cardentey Fundora desarroll\'o satisfactoriamente el trabajo de
    diploma titulado ``Visualizaci\'on Inteligente Autom\'atica de Datos''. En este trabajo el
    estudiante propuso un sistema para la recomendaci\'on de configuraciones gr\'aficas basada en el
    aprendizaje de m\'aquinas sobre grafos.

    Su propuesta se basa, en esencia, en la adaptaci\'on de m\'etodos transductivos de \textit{Knowledge Graph Embedding}
    al contexto de la recomendaci\'on de visualizaciones a trav\'es de aplicar la metaheur\'istica de universos paralelos. Para 
    mostrar la viabilidad de la propuesta, se realizaron un conjunto de experimentos a trav\'es de los cuales se evidencian las posibilidades
    de la propuesta as\'i como sus limitaciones.

    Para poder afrontar el trabajo, el estudiante tuvo que revisar la literatura cient\'ifica relacionada con la
    tem\'atica as\'i como soluciones existentes y bibliotecas de software que pudieran ser apropiadas para su utilizaci\'on.
    Todo ello con sentido cr\'itico, determinando las mejores aproximaciones y tambi\'en las dificultades que presentan.

    Todo el trabajo fue realizado por el estudiante con elevada constancia, capacidad de trabajo y habilidades, tanto de gesti\'on,
    como de desarrollo e investigaci\'on.
    Por estas razones pedimos que le sea otorgada al estudiante Victor Manuel Cardentey Fundora la m\'axima calificaci\'on posible
    y, de esta manera, pueda obtener el t\'itulo de Licenciado en Ciencia de la Computaci\'on.

    \begin{center}
        Dr. Yudivi\'an Almeida Cruz\\
        Facultad de Matem\'atica y Computaci\'on\\
        Universidad de La Habana\\
    \end{center}

    \begin{center}
        Lic. Dayany Alfaro Gonz\'alez\\
        Facultad de Matem\'atica y Computaci\'on\\
        Universidad de La Habana\\
    \end{center}
   
   
\end{opinion}